\begin{otherlanguage}{english}
    \chapter*{Abstract}
    \addcontentsline{toc}{chapter}{Abstract} % si queremos que aparezca en el índice
    %\markboth{ABSTRACT}{ABSTRACT} % encabezado
  
    This project involves the integration of the air conditioning and refrigeration equipment of a supermarket into the kiconex IoT platform.

    Having this type of network allows the collection of data with which to know the behavior of the different equipment in the installation under different conditions, in addition to being able to perform a remote management. That is why kiconex is another step in the technological innovation of any company dedicated to the cold and climate sector, marking a clear difference from its competition.
    
    In this project is also carried out the development of a new wireless communication device, which acts as a gateway from Modbus TCP frames transmitted via WiFi, to Modbus RTU frames via RS485. A kiconex device of this type facilitates equipment integration by avoiding the use of cables and reducing labor with a cleaner and faster installation. For this development, Olimex brand hardware is used, based on the ESP32 chip.
    
    Finally, the control of the Air Handling Unit of the establishment has been programmed, through the ISaGRAF software, which uses several languages of the IEC61131 standard. 

    \vspace{10mm}

    \begin{center}
    \textbf{\Large{Keywords}} \\
    \vspace{10mm}
    kiconex, Modbus, ESP32, iPro, Dixell, Olimex 
    \end{center}
   \end{otherlanguage}
\clearpage