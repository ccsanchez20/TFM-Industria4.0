\chapter*{Resumen}
\addcontentsline{toc}{chapter}{Resumen} % si queremos que aparezca en el índice
%\markboth{RESUMEN}{RESUMEN} % encabezado

% Aquí viene un resumen del proyecto.
En el presente proyecto se realiza la integración de los equipos de climatización y refrigeración de un supermercado en la plataforma IoT de kiconex.

Disponer de una red de este tipo permite la recolección de datos con los que conocer el comportamiento de los distintos equipos de la instalación frente a distintas condiciones, además de poder realizar la gestión remota de la misma. Es por ello que kiconex supone un paso más en la innovación tecnológica de cualquier empresa dedicada al sector del frío y clima, marcando una clara diferencia de la misma frente a su competencia.

En este proyecto también se lleva a cabo el desarrollo de un  nuevo dispositivo de comunicación inalámbrica, que actúa como pasarela de las tramas Modbus TCP transmitidas vía WiFi, a tramas Modbus RTU vía RS485. Un dispositivo kiconex de este tipo facilita la integración de equipos, al evitar el uso de cables y disminuir la mano de obra con una instalación más limpia y rápida. Para este desarrollo se emplea hardware de la marca Olimex, basado en el chip ESP32.

Por último, se ha programado el control de la Unidad de Tratamiento de Aire del establecimiento, a través del software ISaGRAF, que emplea varios lenguajes de la norma IEC61131. 



\vspace{10mm}

\begin{center}
\textbf{\Large{Palabras clave}} \\
\vspace{10mm}
kiconex, Modbus, ESP32, iPro, Dixell, Olimex 
\end{center}