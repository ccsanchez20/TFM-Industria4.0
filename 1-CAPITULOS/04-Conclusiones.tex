%%%%%%%%%%%%%%%%%%%%%%%%%%%%%%%%%%%%%%%%%%%%%%%%%%%%%%%%%%%%%%%%%%%%%%%%%%%%%%%
%% 3.- Pruebas y Resultados
%%%%%%%%%%%%%%%%%%%%%%%%%%%%%%%%%%%%%%%%%%%%%%%%%%%%%%%%%%%%%%%%%%%%%%%%%%%%%%%

\cleardoublepage
\chapter{Conclusiones y Trabajos Futuros}
\chaptermark{Conclusiones y Trabajos Futuros}

\label{chap:conclusiones} % etiqueta para poder referenciar luego en el texto con ~\ref{sec:intro}
% \addcontentsline{toc}{chapter}{Introducción, Objetivos, Metodología y Planificación}

En este Trabajo Fin de Máster, se desarrolla una red basada en kiconex para un supermercado. Una parte de los equipos del establecimiento se han instalado siguiendo la estructura estándar de una red kiconex, sin embargo, para determinados equipos, se ha innovado a través del desarrollo de un nuevo hardware inalámbrico, bautizado con el nombre de kiwi. Además, para la Unidad de Tratamiento de Aire del supermercado, también se ha programado un control a medida, siguiendo las especificaciones de funcionamiento del cliente. Todo esto ha permitido cumplir con los objetivos que se plantearon al comienzo del documento, obteniendo una instalación que reúne las siguientes características:

\begin{itemize}
    \item Monitorización de valores en la plataforma IoT: temperaturas, estados dispositivos, alarmas, etc.
    \item Visualización de valores en el tiempo, en forma de gráficas.
    \item Control remoto de dispositivos: ON/OFF, reinicio de alarmas, etc.
    \item Cambios de consigna y configuración de parámetros de forma remota.
    \item Alertas frente a alarmas.
    \item Funcionamiento programado a través de reglas.
\end{itemize}

En el proceso de desarrollo se ha presentado la plataforma IoT de kiconex, donde se crea y visualiza la instalación. Además, se ha hecho una Introducción a los entornos de programación de ISaGRAF y Visoprog, mediante los cuales se programa el control iPro de Dixell.

% Pruebas realizadas

Se han llevado ha cabo las pruebas necesarias para verificar el correcto funcionamiento del kiwi, y en este proceso también se ha puesto de manifiesto la necesidad de una correcta configuración de red, haciendo indispensable la presencia de personal con conocimientos de redes en la puesta en marcha de cualquier instalación que incluya un dispositivo como este.

Se ha verificado el funcionamiento del software de la UTA sobre un simulador físico real, compuesto del mismo control iPro con su pantalla. Esto asegura el correcto funcionamiento cuando se carga el programa en el control final del cliente.

Finalmente, como se ha explicado en el \hyperref[sec:altaInstalacion]{capítulo 3}, la puesta en marcha ha sido rápida, tras configurar en cada elemento del supermercado su correspondiente dirección Modbus. En dicha puesta en marcha, los elementos tenidos en cuenta en la conectividad con la plataforma IoT son:

\begin{itemize}
    \item En caso de equipos Modbus RTU: 
    \begin{itemize}
        \item Resistencias de terminación.
        \item Conexión de equipos en cadena.
    \end{itemize}
    \item En caso de equipos Modbus TCP: todos los equipos en el mismo rango de IPs que el kibox.
    \item Configuración de la red del kibox: apertura de los puertos de salida necesarios.
\end{itemize}

Todo el proyecto ha seguido la filosofía de funcionamiento de una red con kiconex. El nuevo módulo kiwi programado supone un gran avance en la estructura de dicha red y da la posibilidad de llegar a clientes con otras necesidades. En este caso, como desarrollo futuro, se podría adaptar para gestionar equipos Modbus RTU (a través del conversor RS485) al mismo tiempo que equipos Modbus TCP (a través del puerto Ethernet).

Por otro lado, el programa realizado para el kiwi se apoya en librerías existentes para el chip ESP32. Para dichas librerías, se han detectado en este proyecto problemas de funcionamiento, por lo que, a modo de trabajo futuro, se podría trabajar sobre sus códigos para solucionarlos. Estos problemas de funcionamiento no han afectado al desarrollo del programa del kiwi dado que se pueden paliar mediante otros métodos más rápidos y que han resultado eficaces. En concreto, los problemas están en el paso de un modo de conexión a otro (de modo AP a modo cliente, por ejemplo), y en la desconexión y reconexión WiFi, es decir, la primera configuración de red se realiza sin problemas, pero no es así cuando se desea cambiar a una configuración distinta partiendo de una anterior.

% Trabajos futuros xxx



