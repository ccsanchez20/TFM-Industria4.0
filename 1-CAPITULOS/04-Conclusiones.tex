%%%%%%%%%%%%%%%%%%%%%%%%%%%%%%%%%%%%%%%%%%%%%%%%%%%%%%%%%%%%%%%%%%%%%%%%%%%%%%%
%% 3.- Pruebas y Resultados
%%%%%%%%%%%%%%%%%%%%%%%%%%%%%%%%%%%%%%%%%%%%%%%%%%%%%%%%%%%%%%%%%%%%%%%%%%%%%%%

\cleardoublepage
\chapter{Conclusiones y Trabajos Futuros}
\chaptermark{Conclusiones y Trabajos Futuros}

\label{chap:conclusiones} % etiqueta para poder referenciar luego en el texto con ~\ref{sec:intro}
% \addcontentsline{toc}{chapter}{Introducción, Objetivos, Metodología y Planificación}

Tras realizar el programa y el diseño de la pantalla para el control de la UTA, estos se cargan en un simulador que se compone de un iPro y una pantalla para pruebas. Dado que el programa se ha simulado previamente en el software ISaGRAF, no se han encontrado problemas de funcionamiento en él, sin embargo, es habitual y así ha sido en este caso, encontrar errores de diseño en la pantalla, que se han pasado por alto durante su creación. Estos problemas se han corregido desde el software Visoprog y se han vuelto a cargar.

Cuando el resultado de las pruebas es bueno, desde fábrica realizan un cuadro nuevo con iPro, al que se le carga el programa realizado tras comprobar que el iPro funciona correctamente. Este cuadro se envía al cliente, en este caso el instalador al cargo del supermercado.

