%%%%%%%%%%%%%%%%%%%%%%%%%%%%%%%%%%%%%%%%%%%%%%%%%%%%%%%%%%%%%%%%%%%%%%%%%%%%%%%
%% 2.- DESARROLLO DEL PROYECTO
%%%%%%%%%%%%%%%%%%%%%%%%%%%%%%%%%%%%%%%%%%%%%%%%%%%%%%%%%%%%%%%%%%%%%%%%%%%%%%%

\cleardoublepage
\chapter{Desarrollo del proyecto}
\chaptermark{Desarrollo del proyecto}

\label{chap:desarrolloProyecto} % etiqueta para poder referenciar luego en el texto con ~\ref{sec:intro}
% \addcontentsline{toc}{chapter}{Introducción, Objetivos, Metodología y Planificación

En las siguientes secciones se detalla de forma estructurada el desarrollo de cada una de las partes que componen este proyecto.


\section{Programación iPro}
\label{sec:programacionipro}
Como ya se ha explicado en los capítulos anteriores, el destino del iPro es la programación de varias UTAs existentes en el supermercado. En e Anexo xxx se describe detalladamente qué es una UTA y de qué elementos se puede componer. En el caso del supermercado, la UTA tiene una serie de características que van a simplificar su control:

\begin{itemize}
    \item Ventiladores
\end{itemize}



\section{Diseño Pantalla para el control}
\label{sec:programacionpantalla}
indican las conclusiones generales obtenidas a partir del estudio realizado, se proponen lineas de trabajo futuro, y se indican las publicaciones asociadas al trabajo realizado en este TFM.
\section{Elaboración de librerías en Kiconex}
\label{sec:librerias}
indican las conclusiones generales obtenidas a partir del estudio realizado, se proponen lineas de trabajo futuro, y se indican las publicaciones asociadas al trabajo realizado en este TFM.
\section{Programación dispositivo wireless}
\label{sec:programacionesp32}
indican las conclusiones generales obtenidas a partir del estudio realizado, se proponen lineas de trabajo futuro, y se indican las publicaciones asociadas al trabajo realizado en este TFM.