%%%%%%%%%%%%%%%%%%%%%%%%%%%%%%%%%%%%%%%%%%%%%%%%%%%%%%%%%%%%%%%%%%%%%%%%%%%%%%%
%% 2.- DESARROLLO DEL PROYECTO
%%%%%%%%%%%%%%%%%%%%%%%%%%%%%%%%%%%%%%%%%%%%%%%%%%%%%%%%%%%%%%%%%%%%%%%%%%%%%%%

\cleardoublepage
\chapter{Desarrollo del proyecto}
\chaptermark{Desarrollo del proyecto}

\label{chap:desarrolloProyecto} % etiqueta para poder referenciar luego en el texto con ~\ref{sec:intro}
% \addcontentsline{toc}{chapter}{Introducción, Objetivos, Metodología y Planificación

Capítulo de desarrollo de cada uno de los apartados del proyecto.


\section{Programación iPro}
\label{sec:programacionipro}
Como ya se ha explicado en los capítulos anteriores, el destino del iPro es la programación de varias UTAs existentes en el supermercado. Existen dos tipos de UTA, la que solo coge el aire de la estancia para volver a introducirlo a la temperatura deseada, y la que tiene un sistema de renovación de aire: conduce el aire de la estancia al exterior e introduce el aire desde el exterior, calentándolo o enfriándolo primero. La figura xxx representa ambos sistemas.

El objetivo es realizar un programa estándar que se pueda usar para ambos tipos de UTA, para así evitar realizar nuevos programas en futuros proyectos que incluyan una UTA. Para ello, en la tabla xxx se han recogido las diferencias técnicas entre ambas, para saber hasta que punto se puede adaptar el programa a cualquier caso.

\section{Diseño Pantalla para el control}
\label{sec:programacionpantalla}
indican las conclusiones generales obtenidas a partir del estudio realizado, se proponen lineas de trabajo futuro, y se indican las publicaciones asociadas al trabajo realizado en este TFM.
\section{Elaboración de librerías en Kiconex}
\label{sec:librerias}
indican las conclusiones generales obtenidas a partir del estudio realizado, se proponen lineas de trabajo futuro, y se indican las publicaciones asociadas al trabajo realizado en este TFM.
\section{Programación dispositivo wireless}
\label{sec:programacionesp32}
indican las conclusiones generales obtenidas a partir del estudio realizado, se proponen lineas de trabajo futuro, y se indican las publicaciones asociadas al trabajo realizado en este TFM.