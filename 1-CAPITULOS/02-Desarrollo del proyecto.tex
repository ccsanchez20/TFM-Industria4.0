%%%%%%%%%%%%%%%%%%%%%%%%%%%%%%%%%%%%%%%%%%%%%%%%%%%%%%%%%%%%%%%%%%%%%%%%%%%%%%%
%% 2.- DESARROLLO DEL PROYECTO
%%%%%%%%%%%%%%%%%%%%%%%%%%%%%%%%%%%%%%%%%%%%%%%%%%%%%%%%%%%%%%%%%%%%%%%%%%%%%%%

\cleardoublepage
\chapter{Desarrollo del proyecto}
\chaptermark{Desarrollo del proyecto}

\label{chap:desarrolloProyecto} % etiqueta para poder referenciar luego en el texto con ~\ref{sec:intro}
% \addcontentsline{toc}{chapter}{Introducción, Objetivos, Metodología y Planificación

En las siguientes secciones se detalla de forma estructurada el desarrollo de cada una de las partes que componen este proyecto.


\section{Programación iPro}
\label{sec:programacionipro}
Como ya se ha explicado en los capítulos anteriores, el destino del iPro es la programación de la UTA existente en el supermercado. En el \hyperref[chap:anexoUTA]{Anexo~\ref{chap:anexoUTA}} se describe detalladamente qué es una UTA y todos los elementos de los que se puede componer. Para el caso que nos atañe, la \hyperref[tab:especificacionesUTA]{Tabla~\ref{tab:especificacionesUTA}} recoge las especificaciones concretas del control:

\begin{table}[h]
    %\centering
    \begin{center}
      \setlength\arrayrulewidth{2pt}
      \resizebox{\linewidth}{!}{\begin{tabular}{ | c | c c c c | }
        %\Cline{2pt}{2-5}
        \hhline{|*{5}{-}}
        \cellcolor{lightgray}\textbf{GRUPO} & \multicolumn{1}{c|}{\cellcolor{lightgray}\textbf{Entrada analógica}} & \multicolumn{1}{c|}{\cellcolor{lightgray}\textbf{Salida analógica}} & \multicolumn{1}{c|}{\cellcolor{lightgray}\textbf{Entrada digital}} & \cellcolor{lightgray}\textbf{Salida digital}   \\ \hline
        \footnotesize{\textbf{Ventiladores}} &  & \parbox[c][1.8cm]{0.2\linewidth}{\centering\footnotesize{Ventilador impulsión\\Ventilador retorno}}  & \parbox[c][2.8cm]{0.2\linewidth}{\centering\footnotesize{Seguridad Ventilador impulsión\\Seguridad Ventilador retorno}} &  \\ \hline
        \footnotesize{\textbf{Filtros}} & & & \footnotesize{\parbox[c][2.3cm]{0.2\linewidth}{\centering{Filtro entrada\\Filtro salida\\Filtro retorno}}} & \\ \hline
        \footnotesize{\textbf{Humectador}} & \footnotesize{\parbox[c][1.5cm]{0.2\linewidth}{\centering{Humedad de retorno\\Humedad exterior}}} & & \centering\footnotesize{Indicador nivel agua} & \footnotesize{ON/OFF Humectador} \\ \hline
        \footnotesize{\textbf{Intercambiador de calor}} & \footnotesize{\parbox[c][1.5cm]{0.2\linewidth}{\centering{Temp. impulsión agua\\Temp. retorno agua}}} & & & \footnotesize{Válvula de agua}\\ \hline
        \footnotesize{\textbf{E/S de aire}} & \footnotesize{\parbox[c][3cm]{0.2\linewidth}{\centering{Temp. impulsión aire\\Temp. retorno aire\\Temp. extracción aire\\Temp. aire exterior}}} & & & \\ \hline

      \end{tabular}}
      \caption{Especificaciones E/S UTA.}
      \label{tab:especificacionesUTA}
    \end{center}
  \end{table}

  


\section{Diseño Pantalla para el control}
\label{sec:programacionpantalla}
indican las conclusiones generales obtenidas a partir del estudio realizado, se proponen lineas de trabajo futuro, y se indican las publicaciones asociadas al trabajo realizado en este TFM.
\section{Elaboración de librerías en Kiconex}
\label{sec:librerias}
indican las conclusiones generales obtenidas a partir del estudio realizado, se proponen lineas de trabajo futuro, y se indican las publicaciones asociadas al trabajo realizado en este TFM.
\section{Programación dispositivo wireless}
\label{sec:programacionesp32}
indican las conclusiones generales obtenidas a partir del estudio realizado, se proponen lineas de trabajo futuro, y se indican las publicaciones asociadas al trabajo realizado en este TFM.